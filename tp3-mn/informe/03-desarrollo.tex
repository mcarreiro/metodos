\section{Desarrollo}

\subsection{Vecinos}

\subsection{Bilineal}

\subsection{Direccional}

Para este algoritmo nos basamos en lo explicado por Burden y Faires$[1]$ para el caculo de los splines y en lo desarrollado por Ron Kimmel$[2]$ para el algoritmo en sí. El método lo aplicamos sólo para el color verde mientras que para los otros utilizamos bilineal. 

Lo que hace es:
\begin{lstlisting}[frame=single] 
Para cada celda:
	Interpolo mediante spline su fila y columna
	Calculo sus derivadas aproximadas en direcci\'on horizontal y vertical
	Si la derivada horizontal es mayor
		celda.verde = interpolaci\'on horizontal * 0.3 +  interpolaci\'on vertical *  0.7
	sino
		celda.verde = interpolaci\'on horizontal * 0.3 +  interpolaci\'on vertical * 0.7	
\end{lstlisting}

Dado que las interpolaciones mediante splines no son nada triviales lo explicaremos mas adelante en detalle. Las derivadas en $x$  la aproximamos haciendo $|G(x-1,y)-G(x+1,y)|$ donde G es el valor del color verde en ese punto, la derivada en $y$ es análoga. Dado que un mayor valor en la derivada puede estar indicándonos un potencial borde le damos mayor peso a la derivada cuyo valor es más chico multiplicando a este por 0.7 y a la otra por 0.3. Finalmente las sumamos para obtener el verde correspondiente en nuestra celda.

\subsection{High Quality}

El algoritmo de demosaicing que denominamos "High Quality" se basa en el paper de Malvar, He y Cutler. Su principal atractivo es que realiza una corrección de la imagen luego de aplicar el algoritmo de Interpolación Bilineal pero teniendo en cuenta para calcular un color a los otros dos, en diferencia a los otros algoritmos que solo tienen en cuenta los valores cercanos con respecto a su mismo color.


Para el algoritmo highquality decidimos solamente procesar el color verde de los pixeles ya que solamente medimos la calidad de ese color en el tp.

- Primero hacemos bilineal sobre todos los colores de la imagen
- Por cada pixel de la imagen (exceptuando los bordes):
		
	


