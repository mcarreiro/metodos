\section{Desarrollo}

\subsection{Vecinos}

\subsection{Bilineal}

\subsection{Direccional}

\subsection{High Quality}

Todos los algoritmos anteriores aproximaban el valor mediante un color, pero esto no era suficiente para darle definición ya que muchas veces los colores tienen un brillo o luz que impacta en todos, por lo tanto... 

El algoritmo de demosaicing que denominamos "High Quality" se basa en el paper de Malvar, He y Cutler. Su principal atractivo es que realiza una corrección de la imagen luego de aplicar el algoritmo de Interpolación Bilineal pero teniendo en cuenta para calcular un color a los otros dos, en diferencia a los otros algoritmos que solo tienen en cuenta los valores cercanos con respecto a su mismo color.



Para el algoritmo highquality decidimos solamente procesar el color verde de los pixeles ya que solamente medimos la calidad de ese color en el tp.

- Primero hacemos bilineal sobre todos los colores de la imagen
- Por cada pixel de la imagen (exceptuando los bordes):
	- Si la imagen cae en verde la ignoro
	- Si cae en rojo o azul:
		- Al color verde de ese pixel le sumo el valor del color actual calculado en la bilineal
		- 
		
	


