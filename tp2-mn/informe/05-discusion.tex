\section{Discusi\'on}

\subsection{PageRank}
Claramente podemos notar que a medida que el C crece, el algoritmo toma más iteraciones en achicar la norma. Esto se debe a que el grado de aleatoriedad elimina el peso de la unión entre los sitios e indica una uniformidad en el comportamiento, entonces la matriz si bien estocástica ahora se encuentra distribuida esa suma $=$ 1 por columna en varias filas. Esto produce mayor cantidad de iteraciones en el método de la potencia ya que la mayor uniformidad de la matriz provoca que ninguna 'zona' de la matriz absorba más que las demás.   $[1]$\\
También es bastante notorio que a pesar de que los distintos casos de prueba sean muy diferentes entre si y hasta cientos de veces más grandes, la evolución de la norma converge de formas casi idénticas y lo mismo sucede para las iteraciones requeridas hasta llegar a la norma variando el parámetro c.



\subsection{HITS}
En todos los casos podemos observar que tanto el vector de hubs como el de autridades convergen de forma muy similar, sólo en la instancia grande hay una pequeña diferencia pero es bastante despreciable. 
Por otro lado podemos ver que los casos en los que mas drásitca es la convergencia (abortion y genetic) los valores inciales de la norma manhattan son muy altos (alrrededor de 100), provocando asi que se equiparen con las que comienzan en valores mas bajos pero convergen mas lentamente (movies y standford).
En estos dos últimos casos además podemos notar grandes saltos de convergencia pasando en pocas iteracion de 1$e^{20}$ a menos de 1$e^{80}$, entiendiendo, aca sí, que la diferencia es totalmente despreciable y el valor obtenido ya ha convergido. De todas formas consideramos que puede ser un punto de interés para analizar mas en profundidad ya que más allá de decir que entendemos de eso, no sabríamos explicar porque se produce ese salto. 

\subsection{Ejemplos de comportamiento esperado}

A continuación veremos en redes pequeñas como se comporta cada algoritmo para ver si su comportamiento es el esperado.

\subsubsection{PageRank}
 \begin{figure}[!htb]
\begin{center}
    \includegraphics[scale=0.5]{imagenes/test5.png}
    \caption{Red de 11 nodos, c=0.85}
    \end{center}
\end{figure}
\subsubsection{HITS}
 \begin{figure}[!htb]
\begin{center}
    \includegraphics[scale=0.5]{imagenes/test4.png}
    \caption{Red de 7 nodos}
    \end{center}
\end{figure}

Resultado obtenido:
   $$ 
\begin{bmatrix}
              &    Autoridad  &  Hub \\
 Nodo 1 &   0.000000    &      0.383092       \\
 Nodo 2   &  0.967054    &  0.000000     \\
 Nodo 3   &  0.000000   &     0.383092  \\
 Nodo 4   &  0.180008    &     0.454401       \\
 Nodo 5   &  0.180008    &     0.454401        \\
 Nodo 6   &  0.000000    &      0.383092     \\
 Nodo 7   &  0.000000   &     0.383092 \\
\end{bmatrix} 
$$

Efectivamente podemos observar que en la columna de autoridades el nodo 2 es el mayor ya que es el que mas apuntado esta y todos aquellos que tienen 0 es porque no son apuntados por ninguno. Por otro lado en la columna de hubs podemos ver que los nodos 4 y 5 son los que mayor valor tienen ya que son los que mas apuntan a otros nodos con 2 salidas.

\subsubsection{Indeg}
 	Veamos el comportamiento dada esa pequeña red.

 	Resultado obtenido:
   $$ 
\begin{bmatrix}
              &    Puntaje \\
 Nodo 1 &    0.000000 \\
 Nodo 2   &  0.714285 \\
 Nodo 3   &  0.000000 \\
 Nodo 4   &  0.142857 \\
 Nodo 5   &  0.142857 \\
 Nodo 6   &  0.000000 \\
 Nodo 7   &  0.000000 \\
\end{bmatrix} 
$$

Este algoritmo naive es bastante claro de interpretar, y el resultado es claramente el esperado. El nodo 2 posee mucha más calidad de sitio ya que es el más apuntado, y en el segundo puesto empatando el nodo 4 y 5, por ser aquellos con más sitios apuntándolos, excpetuando el nodo 2. El resto de los nodos no reciben ningún tipo de link hacia ellos, por lo que su puntaje es de cero.

\subsection{Análisis cualitativo}

En esta sección procederemos a discutir sobre la calidad de resultados que obtenemos de cada algoritmo y luego los compararemos entre si.\\
Como el objetivo de este trabajo práctico esta enfocado al ranking web que se le asigna a los distintos sitios de internet, consideramos como buenos resultados aquellos que aparecerían en la primer página de los buscadores, es decir, los primero 10 resultados serán los que consideraremos para el análisis.

\subsubsection{PageRank}
Según el paper de Bryan y Leise, quienes proponen el algoritmo, lo más común es que el valor del navegante aleatorio sea de 0.15. Por lo tanto creemos que con este valor es donde aparecerán los mejores resultados, pero también veremos que sucede con valores de 0.5 y 0.85, ya que estos valores indican por un lado que la probabilidad del navegante entre quedarse e irse es equiprobable y por otro lado es el inverso de lo que ellos consideran como el valor más común. En valores de 0 y 1 no tendrían sentido el análisis ya que por un lado daría la matriz original y por el otro una matriz equiprobable.\\
El caso de prueba que utilizaremos es el dado por la cátedra, \textbf{Abortion}, y lo elegimos ya que es un tema bastante discutido donde se pueden encontrar resultados interesantes.


\paragraph{Resultados con un c=0.15}
\begin{enumerate}
\item
 \textbf{No relacionado con el tema}\\
http://www.allexperts.com/about.asp\\
AllExperts.com
\item
http://www.nrlc.org\\
National Right to Life Organization
\item
\textbf{No relacionado con el tema}\\
http://www.phone-soft.com/at/cyber-world/international/o1480i.htm\\
PHONE-SOFT INTERNET DIRECTORY INTERNATIONAL:HERB THERAPY LINKS
\item

http://www.lm.com/~jdehullu\\
Ariadne's Thread: On abortion, affirmative action, hate speech
\item


http://www.plannedparenthood.org\\
Planned Parenthood Federation of America
\item

http://www.gynpages.com\\
Abortion Clinics OnLine
\item

http://www.care-net.org/link.htm\\
CareNet Links
\item

http://www.naral.org\\
NARAL: Abortion and Reproductive Rights: Choice For Women
 \item

http://www.crosswalk.com/ftr/1,,17,00.htm \\
Crosswalk.com Forums - Welcome
 \item

http://www.cais.com/agm/main\\
The Abortion Rights Activist Home Page

\end{enumerate}

\paragraph{Resultados con un c=0.5}
 \begin{enumerate}
 \item http://www.allexperts.com/about.asp\\
AllExperts.com
 \item http://www.nrlc.org\\
National Right to Life Organization
 \item \textbf{No relacionado con el tema}\\
 http://home.about.com\\
About - The Human Internet
 \item \textbf{No relacionado con el tema}\\
http://www.phone-soft.com/at/cyber-world/international/o1480i.htm\\
PHONE-SOFT INTERNET DIRECTORY INTERNATIONAL:HERB THERAPY LINKS
 \item http://www.lm.com/~jdehullu\\
Ariadne's Thread: On abortion, affirmative action, hate speech
 \item http://www.plannedparenthood.org\\
Planned Parenthood Federation of America
 \item http://www.care-net.org/link.htm\\
CareNet Links
 \item http://www.gynpages.com\\
Abortion Clinics OnLine
 \item http://www.marchforlife.org\\
The March For Life Fund Home Page
 \item \textbf{No relacionado con el tema}\\
http://www.jbs.org\\
The John Birch Society
 \end{enumerate}
 
\paragraph{Resultados con un c=0.85}
 \begin{enumerate}

 \item 
 \textbf{No relacionado con el tema}\\
 http://www.jbs.org\\
The John Birch Society
 \item 
 \textbf{No relacionado con el tema}\\
http://home.about.com\\
About - The Human Internet
 \item 
 \textbf{No relacionado con el tema}\\
http://www.allexperts.com/about.asp\\
AllExperts.com
 \item
  \textbf{No relacionado con el tema}\\
http://www.aobs-store.com\\
American Opinion Book Services Online Store
 \item
http://www.nrlc.org\\
National Right to Life Organization
 \item 
 \textbf{No relacionado con el tema}\\
http://www.trimonline.org\\
TRIMonline - Lower Taxes Through Less Government
 \item
http://www.marchforlife.org\\
The March For Life Fund Home Page
 \item 
 \textbf{No relacionado con el tema}\\
http://www.phone-soft.com/at/cyber-world/international/o1480i.htm\\
PHONE-SOFT INTERNET DIRECTORY INTERNATIONAL:HERB THERAPY LINKS
 \item 
\textbf{No relacionado con el tema}\\
http://www.reagan.com\\
The Reagan Information Interchange
 \item 
 \textbf{No relacionado con el tema}\\
http://www.pregnancycenters.org\\
Pregnancy Centers Online
 \end{enumerate}

En base a los resultados se puede ver como a medida que aumenta el $c$ empiezan a aparecer resultados que poco tienen que ver con el tema directamente, ya que puede estar relacionado de alguna forma o no diferenciarse tanto del eje temático.\\
Nos pareció extraño que aparece siempre muy bien posicionado el sitio web All Experts, que nada tiene que ver con el tema de los abortos, por lo tanto decidimos hacer un foco especial en este para ver porque sucedía esto y llegamos a la conclusión que es debido a que el factor mas determinante es que gran cantidad de sitios referidos al tema y a su vez bien posicionados (aunque fuera del top 10) apuntaban al mismo, y por lo tanto le daban bastante peso a All Experts.\\
Sucede algo parecido con otro sitio de venta de software que aparece pero no nos pareció importante su análisis ya que es un claro caso de publicidad paga en anuncios de los sitios que hablan sobre el tema.\\
Aunque se pueda ver que con un $c$ menor los resultados tienen relación con el tema nos pareció que no son lo suficiente buenos como para considerarlos excelente resultados cuando se busca sobre un tema tan discutido como el aborto, esperando quizás más definiciones sobre el tema y luchas por su legalización/penalización. 

\subsubsection{HITS}

\subsubsection{Comparación}

