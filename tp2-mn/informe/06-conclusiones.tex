
\section{Conclusiones}

\subsection{PageRank}

A medida que fuimos investigando y probando el algoritmo del PageRank nos fue quedando cada vez más claro como es que funciona y que se necesita para obtener un buen resultado para un sitio en particular.
Lo que nos pareció interesante es explicar el algoritmo con la siguiente interpretación metafórica:

\begin{figure}[h!]

       \includegraphics[width=0.45\textwidth]{imagenes/pagerank-pipe-goldfish.png}
           \hfill
        \includegraphics[width=0.45\textwidth]{imagenes/ohcrap-goldfish-flowers.png}

\end{figure}

\FloatBarrier

Tomando como al sitio a analizar en cuestión como la pecera, y a otro sitio que tiene un link a nuestro sitio como el balde se puede observar que cuando el $\textit{recurso}$, en este caso el agua, se reparte equitativamente a todos los destinatarios, por lo tanto, si mi pecera es la única que recibe agua voy a obtener más que si tiene otras bocas la canilla con la cual compartir. Esto es lo mismo que sucede en la web y tiene el cuenta el PageRank, cada sitio le distribuye equitativamente una probabilidad a cada salida, cuya suma total es 1. Por lo tanto, me conviene más que me linkee un sitio con pocas salidas que uno con gran cantidad, pero suponiendo que sus respectivos PageRank son similares, ya que mi PageRank también va a depende del de mis entradas, por lo tanto también hay que tener esto en cuenta, ya que es un factor bastante influyente. Por consiguiente, no solo depende la cantidad de sitios que apuntan a si no también el PageRank de cada uno (la  $\textit{calidad}$)

\newpage


\subsection{HITS}

En el gráfico que nos muestra el tiempo de computo en función del tamaño de los grafos podemos observar que para grafos grandes este algoritmo se vuelve bastante ineficiente. Sin embargo no debemos olvidar que en su paper$[2]$ Kleinberg habla de que este algoritmo debe ser aplicado no sobre toda la red sin sobre un subconjunto de la misma ($\textit{root set}$) obtenido de una busqueda incial. Por lo tanto si acotamos el análisis a los grafos mas acotados podemos ver que el tiempo de computo es aceptable y hasta muy parecido al de page rank. 

\subsection{INDEG}

Este algoritmo es bastante simple y en una red chica y confiable puede llegar a valer. Igualmente tiene mucho peso la confiabilidad, ya que es muy simple de crecer tu puntaje, simplemente creando páginas tontas que apunten a tus sitios. 