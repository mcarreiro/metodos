\section{Resumen}

Los sitios web a medida que fueron creciendo en cantidad en la época de los 90's, complicó el acceso a ellos o a menos que alguien te comentara o a través de publicidades, era muy dificil conocerlos. Es por eso, que fue el auge de los buscadores que a partir de palabras claves, podrían devolverte sitios que existan que puedan llegar a responder tu pregunta o tener el contenido. Un primer problema de entrada, es que, como todo en la vida, la calidad de dicho contenido puede no ser el deseado y existan mejores. Durante este trabajo repasaremos 3 algoritmos conocidos de ranqueo de páginas web, y veremos los resultados y los compararemos. \\
Una vez que sepamos cómo funcionan y cómo ordenan y ubican en los resultados dichos algoritmos, intentaremos responder a la pregunta: cuáles son los pasos a seguir para poder mejorar tu sitio y que salga dado una red, con mejor puntaje que la competencia.
