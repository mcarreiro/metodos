\section{Resultados}

A continuaci\'on se presentan los resultados obtenidos de utilizar la herramienta implementada para
realizar una escucha pasiva de distintas redes locales y calcular las entrop\'ias de las fuentes 
definidas. Estos experimentos se realizaron en 4 lugares distintos: en el shopping Alto Palermo, en
un McDonald's, en un Subway, y por \'ultimo, en la casa de un miembro del equipo para comparar el 
tr\'afico en una red controlada.

\subsection{Red del Alto Palermo}

Para realizar este experimento, capturamos los paquetes de la red del shopping Alto Palermo. La IP 
del host que capturaba los paquetes es \verb@172.17.82.139@, y la duraci\'on del expermiento fue de 
30 minutos. Para ilustrar la comunicaci\'on que se estableci\'o en la red durante el experimento,
se utiliza un grafo que muestra la interacci\'on entre las IPs con los paquetes que se enviaron. Los
nodos del grafo representan las IPs que participaron en la comunicaci\'on; los ejes son dirigidos, si
existe un eje del nodo $IP_1$ al nodo $IP_2$ significa que $IP_1$ es el emisor de un paquete e 
$IP_2$ es el destinatario de dicho paquete. Adem\'as, los ejes tienen un peso que es la cantidad
de veces que se emitieron paquetes desde una IP a otra. \\

\begin{figure}[htb]
\begin{center}
\includegraphics[scale=0.70]{imagenes/grafo_alto_palermo.png} 
\caption{Grafo de paquetes para la red del Alto Palermo} 
\end{center}
\end{figure}

Para comenzar, se supone que la IP \verb@172.17.0.1@  corresponde a un router asignado como default 
\emph{gateway} para la mayor\'ia de los hosts, por su dominio y dado que nuestro host le realiza 
muchos pedidos y que \'el se comunica con otros hosts. 

\newpage

\begin{figure}[htb]
\begin{center}
\includegraphics[scale=0.70]{imagenes/infodst_alto_palermo.png} 
\caption{Cantidad de informaci\'on de cada IP como destinatario, red Alto Palermo} 
\end{center}
\end{figure}

En el gr\'afico anterior se muestra la cantidad de informaci\'on por cada IP destino. Es decir, la 
cantidad de informaci\'on de cada IP tomando como fuente de informaci\'on $S_{dst}$ en esta red. Notar
que la l\'inea horizontal del gr\'afico corresponde al valor de la entrop\'ia de la fuente.
El gr\'afico sobre la fuente de informaci\'on $S_{src}$ no aporta muchos datos ya que, a partir del 
grafo de comunicaciones se puede ver que a los \'unicos emisores de paquetes son la IP del router y 
la IP del host. En base a esto y como supon\'iamos, se aprecia que la menor cantidad de informaci\'on 
y la \'unica que se ubica por debajo del valor de entrop\'ia (cuyo valor es de 2.15) está dada por 
la IP asignada al router.

\subsection{Red del Subway}

En este caso es realiz\'o la captura de paquetes de la red de un Subway. Un aspecto a observar es
que la red \emph{wireless} (el Wi-Fi) no funcionaba o no ten\'ia sen\~al, as\'i que los paquetes ARP
capturados est\'an en el contexto de Ethernet. A continuaci\'on se muestra el gr\'afico 
correspondiente a la cantidad de informaci\'on de cada IP de la fuente $S_{src}$, en una captura
de una duraci\'on de 30 minutos. Nuevamente, la l\'inea roja se refiere al valor de la entrop\'ia 
de la fuente (cuyo valor es 0.4448).
  
\begin{figure}[htb]
\begin{center}
\includegraphics[scale=0.42]{imagenes/Informacion_Subway.png}
\caption{Cantidad de informaci\'on de cada IP como emisor, red Subway}
\end{center}
\end{figure}

En este caso, el gr\'afico respecto a la fuente $S_dst$ no tiene mucha relevancia porque el 
destinatario era la IP \verb@192.168.1.1@ con una alt\'isima probabilidad. Es decir, esa IP 
correspond\'ia al router de la red analizada. 

\subsection{Red de Casa}

Los paquetes recolectados de esta red representan una captura de 30 minutos aproximadamente de los 
paquetes ARP \emph{who-has} que fueron enviados por broadcast a trav\'es de la red Wi-Fi de la casa 
de uno de los miembros del equipo (vive en un departamento). EL gr\'afico corresponde a la cantidad
de informaci\'on de cada IP de la fuente $S_{dst}$. Recordar que la l\'inea horizontal roja es el
valor de la entrop\'ia de la fuente.

\begin{figure}[htb]
\begin{center}
\includegraphics[scale=0.42]{imagenes/Informacion_Casa.png}
\caption{Cantidad de informaci\'on de cada IP como destinatario, red Casa}
\end{center}
\end{figure}

\subsection{Red del McDonald's}

La herramienta implementada tambi\'en se ejecuto en un McDonald's para observar el tr\'afico de 
paquetes que tiene. Este experimento fue el m\'as corto: se dej\'o corriedo la herramienta 15 minutos.
En este experimento se observ\'o una alta actividad de red, es decir, paquetes enviados en casi
todo momento. A continuaci\'on se muestra el gr\'afico de la cantidad de informaci\'on de cada IP
tomando como fuente $S_{dst}$. Recordar que la l\'inea horizontal roja es el valor de la entrop\'ia 
de la fuente.

\begin{figure}[htb]
\begin{center}
\includegraphics[scale=0.40]{imagenes/Informacion_Mc.png}
\caption{Cantidad de informaci\'on de cada IP como destinatario, red McDonald's}
\end{center}
\end{figure}

Lamentablemente, por la gran cantidad de IPs que participaron en la comunicaci\'on, en el gr\'afico
no se pueden ver con claridad las IPs pero en la secci\'on de discusi\'on se hablar\'a de las IPs
que tienen relevancia para nosotros.

Adem\'as se hizo un grafo reducido de la comunicaci\'on entre las IPs de la red, para ver algunos
aspectos interesantes de la red. El grafo tiene el mismo estilo que el confeccionado para la red 
del Alto Palermo: cada nodo representa una IP y existe un eje dirigido de una IP hacia otra si hubo
un env\'io.

\begin{figure}[htb]
\begin{center}
\includegraphics[scale=0.25]{imagenes/grafo1.png}
\caption{Grafo reducido de paquetes para la red del McDonald's}
\end{center}
\end{figure}

Un aspecto interesante que se ve r\'apido en este grafo es que existen pedidos ARP donde el emisor
y el destinatario son la misma IP (por ejemplo, \verb@172.17.203.241@ o \verb@172.17.203.82@). 
Adem\'as se observa que el grueso de pedidos va a parar al dispositivo que tiene asignada la IP
\verb@172.17.203.1@, es decir, el router que oficia de \emph{gateway}.


