\section{Resultados}


\subsection{Resultados Random}

A continuación observaremos una matriz en la que la única sanguijuela que se debe eliminar es la del centro. Esta se encuentra exactamente en el punto central. Adicionalmente hay otras 4 sanguijuelas en las puntas. Matar a estas no soluciona nada ya que la del centro esta aplicando una temperatura constante de 400 Cº. 

\begin{figure}[htb]
\begin{center}
\includegraphics[scale=0.70]{imagenes/test5.png} 
\caption{Parabrisas con 5 sanguijuelas} 
\end{center}
\end{figure}


Ahora veamos que obtenemos al aplicarle distintas veces el algoritmo de solución random explicado en el desarrollo (\ref{sec:solucionRandom}).
\newpage

\begin{figure}[htb]
\begin{center}
\includegraphics[scale=0.50]{imagenes/random_1.png} 
\caption{Resultado primer corrida} 
\end{center}
\end{figure}


\begin{figure}[htb]
\begin{center}
\includegraphics[scale=0.50]{imagenes/random_2.png} 
\caption{Resultado segunda corrida} 
\end{center}
\end{figure}
\newpage

\begin{figure}[htb]
\begin{center}
\includegraphics[scale=0.50]{imagenes/random_3.png} 
\caption{Resultado tercer corrida} 
\end{center}
\end{figure}

Como podemos observar la solución óptima (la que menos sanguijuelas elimina), es la segunda. Pero como esta solución es completamente random en la primera corrida mata 2 sanguijuelas antes de elegir la correcta y en la tercera 4. Sólo en la segunda elije en el primer intento la sanguijuela correcta.
