\section{Conclusiones}

\subsection{Random vs Greedy}

Como observamos en los resultados para el caso en el que la sanguijuela/s se encuentran en el medio claramente es mejor el greedy, ya que siempre mata a la sanguijuela correcta a diferencia del random que sólo lo hizo en uno de los 3 intentos. Por otro lado si tenemos en cuenta los tiempos que pueden llegar a demorar recalcular las temperaturas, esto le da todavìa una mayor importancia, ya que se reduce considerablemente el tiempo de ejecución.

Por otro lado en el segundo caso, donde es un poco mas conflictivo para la lógica de ir sacando los del medio, no sólo sigue siendo claramente mejor que el random sino que ademàs también da la mejor por solución. Por lo tanto podemos concluir que no sólo es un mejor algoritmo que el random sino que además se comporta de forma óptima para los casos que pensabamos iban a ser mas conflictivos.

\newpage

\section{Referencias}\label{sec:MatrizCuadrada}

\begin{itemize}

\item Una matriz de n por m elementos, es una matriz cuadrada si el número de filas es igual al número columnas, es decir, n = m y se dice, entonces que la matriz es de orden n

\item En matemáticas una matriz se le llama matriz banda cuando es una matriz donde los valores no nulos son confinados en un entorno de la diagonal principal, formando una banda de valores no nulos que completan la diagonal principal de la matriz y más diagonales en cada uno de sus costados.

\item Un algoritmo Greedy (también conocido como ávido, devorador o goloso) es aquel que, para resolver un determinado problema, sigue una heurística consistente en elegir la opción óptima en cada paso local con la esperanza de llegar a una solución general óptima. Este esquema algorítmico es el que menos dificultades plantea a la hora de diseñar y comprobar su funcionamiento.

\end{itemize}






