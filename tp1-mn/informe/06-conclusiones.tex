\section{Conclusiones}

Se pudo observar que capturar los paquetes ARP en distintas redes locales da bastante informaci\'on
sobre la red. Es decir, con un an\'alisis est\'adistico sobre los paquetes ARP \emph{who-has} 
tomando como referencia las fuentes de informaci\'on que se pueden definir a partir de las IPs
que emiten mensajes y las IPs detinatarias, se puede por ejemplo deducir cu\'al es el router de
la red. Otra manera de verlo es que si ya se sabe cu\'al es la IP del router, los experimentos
sirven para ver que efectivamente el router es el dispositivos m\'as solicitado, como se puede
ver en la experimentaci\'on que se realiz\'o en el departamento de un miembro del equipo.

Se ejecut\'o la herramienta implementada en distintas redes, lo que supuso distintos resultados. 
Lo interesante de esto es detectar nodos distinguidos o un comportamiento especial que hizo que
tuvi\'eramos que averiguar ciertas cosas, como sucedi\'o en las capturas de la red del McDonald's
en donde aparece una forma de asignar IPs cuando se levantan interfaces. Asimismo, con los 
histogramas hechos de las fuentes correspondientes se puede ver el grado de actividad de ciertos
dispositivos en la red (por lo menos en el momento en que fue realizado el experimento; para mayor
precisi\'on se deber\'ia realizar una captura m\'as amplia y de m\'as duraci\'on).

Como \'ultima observaci\'on cabe destacar que no se debi\'o participar 'agresivamente' en la red. Es
decir, la escucha de los paquetes ARP fue pasiva. S\'olo con recopilar informaci\'on durante un 
tiempo determinado se puede deducir con bastante confiabilidad qu\'e dispositivo hace de servidor
para el resto de los nodos de la red. 

\newpage

\section{Referencias}\label{sec:MatrizCuadrada}

\begin{itemize}

\item Una matriz de n por m elementos, es una matriz cuadrada si el número de filas es igual al número columnas, es decir, n = m y se dice, entonces que la matriz es de orden n

\item En matemáticas una matriz se le llama matriz banda cuando es una matriz donde los valores no nulos son confinados en un entorno de la diagonal principal, formando una banda de valores no nulos que completan la diagonal principal de la matriz y más diagonales en cada uno de sus costados.

\end{itemize}






