\section{Introducci\'on te\'orica}

Para escuchar una determinada red local, se va a utilizar el protocolo ARP (\emph{Address Resolution 
Protocol}) para capturar los distintos paquetes de este formato enviados a trav\'es de la red. ARP 
es un protocolo que permite mapear direcciones de nivel de red a direcciones f\'isicas. Es decir, 
establece una relaci\'on entre las capa de red y la capa de enlace.

Se necesita un formato de paquete determinado para encapsular este protocolo de resoluci\'on de 
direcciones, para mapear las direcciones de red en direcciones f\'isicas. El paquete posee el 
siguiente formato:

\begin{center}
\includegraphics[scale=0.55]{imagenes/arppacket.png}
\end{center}

Los primeros dos campos indican los protocolos de nivel de enlace y nivel de red involucrados en la
comunicaci\'on. Los siguientes dos campos contienen las longitudes en bytes de cada direcci\'on 
de hardware y de protocolo. El campo \verb@Oper@ puede tomar los valores 1 (\emph{who-has}) o 2 
(\emph{reply}): el c\'odigo \emph{who-has} se refiere a que el emisor del paquete necesita saber
la direcci\'on f\'isica correspondiente a cierto IP; el c\'odigo \emph{reply} permite a un 
dispositivo dar a conocer su direcci\'on de hardware.

Los siguientes campos son las direcciones de red y las direcciones de hardware tanto del dispositivo 
emisor como del dispositivo destinatario de los paquetes. 

Entonces, el objetivo es capturar este tipo de paquetes para analizar la cantidad de informaci\'on
de las distintas IPs de la red que se comunican y calcular la entrop\'ia de fuentes de informaci\'on
basadas en los paquetes ARP. Recordar la f\'ormula para obtener la cantidad de informaci\'on de un 
s\'imbolo $s$:

$I(s) = -log(P(s))$

Y para calcular la entrop\'ia de una fuente de informaci\'on $S$:

$H(S) = \displaystyle \sum_{s\in S} P(s) * I(s)$




