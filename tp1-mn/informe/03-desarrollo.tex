\section{Desarrollo}

Como se mencion\'o anteriormente, el objetivo es analizar la estructura de algunas redes locales
mediante la captura de paquetes ARP que se env\'ian por esas redes. Estos paquetes servir\'an para
obtener informaci\'on sobre los dispositivos de red y sacar conclusiones de los resultados que se 
consigan.

Lo primero que se hace en el trabajo es implementar una herramienta para escuchar de manera pasiva
una red local. La idea es que la red local que analicemos no sea una red controlada, para favorecer
una an\'alisis m\'as rico (si fuera controlada, sabr\'iamos por ejemplo cu\'al es la IP del router).
Para eso se utiliza \emph{Scapy} una biblioteca escrita en \verb@Python@, para la captura y 
manipulaci\'on de paquetes en redes.

El segundo paso es definir los siguientes modelos de fuente de informaci\'on:

\begin{itemize}
\item $S_{src} = \{s_1,...,s_k\}$ donde cada s\'imbolo $s_i$ es una direcci\'on IP que aparece como 
direcci\'on origen en los paquetes ARP de tipo \emph{who-has}. 

\item $S_{dst} = \{d_1,...,d_k\}$ donde cada s\'imbolo $d_i$ es una direcci\'on IP que aparece como 
direcci\'on destino en los paquetes ARP de tipo \emph{who-has}.
\end{itemize}

Con estas fuentes definidas, ya se puede estimar las probabilidades de cada IP que necesitemos de 
los paquetes ARP capturados y calcular la cantidad de informaci\'on de cada uno para luego conocer
la entrop\'ia de cada fuente. Para estimar las probabilidades de las IPs en las fuentes, lo que se 
hace es simple. Supongamos que \verb@#paquetes@ es la cantidad total de paquetes ARP 
\emph{who-has} obtenidos y \verb@#apariciones@ es la cantidad de veces que aparece una 
\emph{ip} determinada (la explicaci\'on sirve para ambas fuentes de informaci\'on), entonces se 
define la probabilidad de esa \emph{ip} como: 

P(\emph{ip}) = $\dfrac{\#apariciones}{\#paquetes}$

Con esta definici\'on se puede calcular la cantidad de informaci\'on de cada IP obtenida en las 
fuentes. Luego, es f\'acil obtener la entrop\'ia de cada fuente de informaci\'on.

Entonces, haciendo uso de esta herramienta implementada, se deben realizar capturas de paquetes ARP
sobre distintas LANs (\emph{Local Area Network}) para poder hallar nodos (dispositivos de red) 
distinguidos. El router que oficia como \emph{gateway} (se comunica con otras redes) de la LAN 
analizada es de particular inter\'es. La probabilidad de que su IP aparezca en un paquete ARP 
\emph{who-has}, sea como emisor o como destinatario deberi\'ia ser alta. Es decir, en teori\'a 
deber\'ia ser la IP m\'as frecuente en este tipo de paquetes porque es el canal comu\'n de 
comunicaci\'on. Otra manera de verlo, es la siguiente: se busca la direcci\'on IP (notar que puede
haber m\'as de un \emph{gateway})cuya probabilidades la m\'as alta y por tanto, es la que menor 
cantidad de informaci\'on aporta. Se supone que las direcciones IP cuya informaci\'on sea m\'as 
cercana al valor de la entrop\'ia de la fuente ser\'an los \emph{gateways} de la red.

Asimismo, adem\'as de analizar la LAN para poder identificar el o los \emph{gateways}, se monitorea
la actividad de las otras IPs que se referencian en la red para registrar la frecuencia de cada 
una, comparando su informaci\'on con la entrop\'ia de la fuente y quiz\'as, encontrar otros nodos
especiales.















