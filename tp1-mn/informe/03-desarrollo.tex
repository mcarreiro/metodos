\section{Desarrollo}

\subsection{Matriz Banda}

Como explicamos en la introducción de la matriz banda, llegamos a la conclusión de que la matriz de ecuaciones de la representación del parabrisas tenía forma de la denominada "matriz banda" cuya respectiva banda era de tamaño m+m+1 con m siendo el ancho del parabrisas. Este es un hecho importante ya que las matrices banda tienen caracteristicas especiales de las cuales es posible la optimización temporal y espacial para resolver el sistema de ecuaciones con eliminación gausseana.

\subsubsection{Optimización espacial}

El problema de representar la matriz original es que la mayoría de los valores son 0 y solo importan los elementos de la banda de la matriz, por lo tanto una forma de optimizar espacialmente es solo guardar la banda, con lo que se logra reducir considerablemente el espacio en esta representación, ya que suponiendo que se tiene un parabrisas de n filas y m columnas, el tamaño de la matriz original sería de $(n*m)^2$, mientras que con la optimización de matriz banda quedaría de tamaño (n*m)*(2*m+1)

\subsubsection{Optimización temporal}
\subsubsection{Ejemplo}
\begin{verbatim}

por cada posición del parabrisas
       pos = fila posicion + columna posicion * ancho;
		
              si en esa posición hay una SANGUIJUELA:
			
                     bandMatrix[pos][ancho] = 1;
                     bandMatrix[pos][res] = ts;
		
              si esa posición es borde FRIO:
                    
                      bandMatrix[pos][ancho] = 1;
                     bandMatrix[pos][res] = -100;


              en caso de que esa posición sea VACIA:
                     bandMatrix[pos][ancho] = -4;
                     bandMatrix[pos][res] = 0;
                    
                    si la izquierda no es vacìa
                            bandMatrix[pos][res] -= matrix[i][j-1]->temp;
                     else
                             bandMatrix[pos][ancho-1] = 1;

                   si la de arriba no es vacìa bandMatrix[pos][res] -= matrix[i-1][j]->temp;
                     else bandMatrix[pos][0] = 1;

                     si la de la derecha no es vacìa
                            bandMatrix[pos][res] -= matrix[i][j+1]->temp;
                     else bandMatrix[pos][ancho+1] = 1;
                     
                     si la de abajo no es vacìa
                      bandMatrix[pos][res] -= matrix[i+1][j]->temp;
                     else bandMatrix[pos][ancho*2] = 1;
		

\end{verbatim}
\begin{figure}[htb]
\begin{center}
\includegraphics[scale=0.70]{imagenes/parabrisasej.png} 
\caption{Vista de la representación de Parabrisas del ejemplo} 
\end{center}
\end{figure}

\newpage
\begin{figure}[htb]
\begin{center}
\includegraphics[scale=0.50]{imagenes/matrizej.png} 
\caption{Matriz de ecuaciones del ejemplo} 
\end{center}
\end{figure}

\newpage

\begin{figure}[htb]
\begin{center}
\includegraphics[scale=0.70]{imagenes/matrizbandaej.png} 
\caption{Matriz banda del ejemplo} 
\end{center}
\end{figure}

\newpage

\subsection{Soluciones}

Para resolver el problema que se nos pide, planteamos distintos tipos de solución. Inicialmente encaramos el problema de manera exacta, pero lo deshechamos inmediatamente ya que sabiendo que es un problema de decisión, el tiempo para determinar cuál es el número mínimo de sanguijuelas a eliminar es exponencial en la cantidad de estas. Es por eso que decidimos realizar una heurística, ya que el tiempo es preciado cuando el parabrisas de nuestra nave tiene poco tiempo antes de que se rompa en caso de estar en situación crítica. Para modo de comparación tomamos dos tipos de algoritmos para establecer que es una mejor solución.



seleccionar que sanguijuelas matar, pensamos una solución Greedy  (\ref{sec:MatrizCuadrada}) + Local Search . Esta consiste, a grandes rasgos, en ir matando las sanguijuelas del medio (ya que son las que mayor temperatura generan en el punto crìtico) hasta que el punto central este por debajo del 235 Cº. Para tener algún parámetro de referencia pensamos en buscar otra solución posible, y asì  compararlas y chequear que esta sea mejor. Esta solución alternativa consiste básicamente en seleccionar aleatoriamente que sanguijuelas aniquilar. Procederemos ahora a explicar en detalle las implementaciones de estas soluciones.

\subsection{Solución Greedy + Local Search}
Esta solución consiste, a grandes rasgos, en ir matando las sanguijuelas del medio (ya que son las que mayor temperatura generan en el punto crítico) hasta que el punto central este por debajo del 235 Cº. Finalmente, tratar de realizar una búsqueda local intentado volcar de vuelta en el parabrisas sanguijuelas que habíamos decidido sacar y viendo si era realmente necesario eliminarla para poder salir del estado crítico. La decisión de hacer LocalSearch está dada para el caso de que exista una sanguijuela en el centro y dos a la misma distancia. Pero de un lado haya una linea de muchas sanguijuelas, por lo que es muy probable que con sacar la del lado de la linea fuera suficiente.

Pseudocodigo:

\begin{verbatim}
Class Windshield{
    greedySolution(){
        orderLeachesByDistanceToCenter()
         while(!this->isCooledDown(){
             leachesToRemove(this->removeFirstNotRemovedLeachOrderedCentrically())
         }

         localSearchTryToReturnLeaches(leachesToRemove);

         return leachesToRemove;
    } 
    isCooledDown(){
         recalculateByBandMatrix();
         return (matrix.centerPoint < Ts)
    }
    orderLeachesByDistanceToCenter(){
        sort(leaches).by(lambda {|leach,otherLeach| leach.distanceToCenter < otherLeach.distanceToCenter})
    }

    localSearchTryToReturnLeaches(leachesToRemove){
        for (leach in leachesToRemove){
            putBackLeachInWindshield(leach);
            if (isCooledDown){
                leachesToRemove.erase(leach)
            }else{
                takeOutLeachFromWindshield(leach);
            }
        }
    }
}
\end{verbatim}



\subsection{Solución Random}\label{sec:solucionRandom}


En esta solución seleccionamos de nuestro array de posiciones de sanguijuelas (que es el array recibido por parametro, osea con las posiciones sin discretizar) una al azar y la eliminamos. Luego ejecutamos de nuevo el cálculo de las temperaturas y chequeamos si el punto crìtico esta por debajo del 235 Cº.Si no lo está, elegimos otra al azar y repetimos el proceso hasta que lo esté. 

Pseudocodigo:

\begin{verbatim}
Class Windshield{
    randomSolution(){
         while(!this->isCooledDown(){
             this->randomKill()
         }
    } 
    isCooledDown(){
         return (matrix.centerPoint < Ts)
    }
    randomKill(){
         randomRemove(posSanguijuelas)
         this->recalculateTemps()
    }
}
\end{verbatim}

Vale aclarar que el recalculateTemps utiliza el metodo band matrix, ya que este es màs rápido.









