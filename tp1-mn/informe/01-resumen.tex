\section{Resumen}

Mediante el manejo de matrices se buscará modelar y solucionar un problema de la realidad.Cómo esta esta compuesta de infinitas variables dicha modelización implicará una inevitable discretización. Es decir trabajar con una cantidad acotada de variables del problema (sólo las relevantes).
Si bien el problema en cuestión consiste en decidir cual/es de las sanguijuelas que están adheridas al parabrisa se debe eliminar la modelización del mismo no es trivial. Es más, podría decirse que este proceso es mucho mas complejo y costoso que la solución en sí. ¿Por qué? porque la creación de la matriz que represente al parabrisa y el cálculo de las temperaturas en cada punto, si no se usa buen método, podría llegar a demorar mucho tiempo, tirar overflows o directamente nunca terminar.

Por esto es que a lo largo de este tp haremos mucho foco en como calcular las temperaturas, como optimizar el espacio ocupado por la matriz obtenida y como optimizar lo mas posible todas las operaciones matriciales.


