\section{Discusi\'on}

En esta secci\'on se presenta un an\'alisis un poco m\'as profundo y las deducciones que se pueden 
extraer de los experimentos realizados. Se separa el an\'alisis por cada red escuchada. 

\subsection{Red del Alto Palermo}

En esta red se observa una gran interacci\'on con el router, cuya IP es \verb@172.17.0.1@. Esto 
sucede porque el router es el dispositivo por el que pasan en general la mayor\'ia de los pedidos.
Un aspecto interesante que se ve en esta comunicaci\'on es que no parece haber un pedido fuera de
la red, es decir, a otro dominio. Lo que se puede deducir es que el dominio de esta red parece ser 
\verb@172.17.0.0 /16@ donde la IP del router es \verb@172.17.0.1@, el cu\'al se comunica con el
resto de los dispositivos bajo este dominio.

\subsection{Red de Subway}

Lo primero que se ve es la baja cantidad de IPs que interactuan en la comunicaci\'on. Adem\'as se
observa que la interacci\'on se realiza en una red privada del dominio \verb@192.168.0.0 /16@ que 
se suele usar para redes dom\'esticas. Naturalmente la IP del router en este tipo de red es 
\verb@192.168.0.1@.

De los datos obtenidos, se observ\'o que casi todos los pedidos ten\'ian como destinatario la IP
del router. Algo que se puede ver en el histograma realizado para esta red es que la IP que menos
informaci\'on tiene es \verb@192.168.1.20@, lo que significa que es la que m\'as pedidos ARP 
efectu\'o.

\subsection{Red de Casa}

La primera observaci\'on que se puede realizar es que a pesar de contar unicamente con 5 diferentes 
dispositivos con acceso a la red (incluyendo el router) se pueden observar un total de 13 diferentes 
IPs. En segunda instancia se puede detectar f\'acilmente la linea de la entrop\'ia en un valor 
aproximado de $2.1$ y as\'i tambi\'en la \'unica IP que aporta menor informaci\'on que la entrop\'ia 
a la IP del router en valor de \verb@192.168.0.1@ el cual da salida hacia internet para los 
dispositivos de la red. Notar que el dominio de la IP del router vuelve hacer la red dom\'estica \verb@192.168.0.0 /16@, lo que es natural porque el experimento fue realizado en un departamento.

\subsection{Red de McDonald's}

Esta red es la que m\'as datos aport\'o al an\'alisis, fue claramente en la que m\'as pedidos ARP
se hicieron entre distintos nodos. Esto hace que tenga algunos aspectos interesantes para observar,
en comparaci\'on con las otras redes analizadas. A diferencia de las anteriores redes se puede 
destacar que al ser una red de mayor incidencia (contar con una cantidad alta de dispositivos 
en la comunicaci\'on), la diferencia de informaci\'on entre la IP de menor informaci\'on (el router) 
y la entrop\'ia de la fuente es mayor.

Lo que no se observa bien en el histograma es que la mayor\'ia de las IPs son del estilo 
\verb@172.17.203.X@, lo que nos induce a pensar que el dominio de la red es \verb@172.17.203.0 \24@
y que la mayor interacci\'on se produce en ese dominio, en donde la mayor\'ia de los pedidos ARP
tienen como destinatario la IP \verb@172.17.203.1@ que se deduce que la IP correspondiente al 
router de la LAN (es la que menos informaci\'on aporta en la fuente graficada en el histograma, es
decir, es la barra m\'as peque\~na). 

Otro aspecto interesante que se observ\'o es la gran cantidad de veces que figuraba \verb@0.0.0.0@
como IP emisora. Este es un pedido ARP gratuito que puede ser utilizado por cualquier dispositivo
para verificar que determinada IP no est\'e siendo usada (la IP destino del pedido). La aparici\'on
de esta IP tiene relaci\'on con el siguiente hecho: la IP destino de estos pedidos siempre 
pertenec\'ian al dominio \verb@169.254.0.0 /16@, usada como \emph{broadcast} de \emph{link local}. Se
usa para comunicaci\'on entre hosts en un s\'olo \emph{link}. Los hosts obtienen direcciones de este
dominio mediante una autoconfiguraci\'on, por ejemplo, en situaciones donde el servidor DHCP 
(\emph{Dynamic Host Configuration Protocol}) no se encuentra. Es decir, cuando una interfaz pierde
la conexi\'on o es activada, si no puede recibir una IP con el protocolo DHCP se le asigna una IP
del dominio \verb@169.254.0.0 /16@. Por lo tanto, con el pedido de parte de \verb@0.0.0.0@ se 
quiere fijar si la IP que le fue asignada no est\'a siendo usada en la red. El \'ultimo comentario
con este dominio, que tambi\'en est\'a relacionado con el pedido gratuito, es el siguiente: se 
observ\'o un pedido ARP donde la IP emisora era la misma que la destinataria por cada IP del dominio \verb@169.254.0.0 /16@. Esta particularidad la vimos en el taller; cuando se levanta una interfaz
muchos sistemas env\'ian autom\'aticamente un pedido ARP gratuito con esta caracter\'istica. De esta
manera se pueden detectar IPs duplicadas (en caso de recibir respuesta) y actualiza	la cach\'e ARP
de los dem\'as hosts de la red.


